\documentclass[
]{jss}

\usepackage[utf8]{inputenc}

\providecommand{\tightlist}{%
  \setlength{\itemsep}{0pt}\setlength{\parskip}{0pt}}

\author{
Sahir Bhatnagar *\\McGill University \And Maxime Turgeon *\\University of Manitoba \AND Jesse Islam\\McGill University \And James Hanley\\McGill University \And Olli Saarela\\University of Toronto
}
\title{\pkg{casebase}: An Alternative Framework For Survival Analysis}

\Plainauthor{Sahir Bhatnagar *, Maxime Turgeon *, Jesse Islam, James Hanley, Olli Saarela}
\Plaintitle{casebase: An Alternative Framework For Survival Analysis}
\Shorttitle{\pkg{casebase}: An Alternative Framework For Survival Analysis}

\Abstract{
The abstract of the article. * joint co-authors
}

\Keywords{keywords, not capitalized, \proglang{Java}}
\Plainkeywords{keywords, not capitalized, Java}

%% publication information
%% \Volume{50}
%% \Issue{9}
%% \Month{June}
%% \Year{2012}
%% \Submitdate{}
%% \Acceptdate{2012-06-04}

\Address{
    Sahir Bhatnagar *\\
  McGill University\\
  1020 Pine Avenue West Montreal, QC, Canada H3A 1A2\\
  E-mail: \email{sahir.bhatnagar@mail.mcgill.ca}\\
  URL: \url{http://sahirbhatnagar.com/}\\~\\
      Maxime Turgeon *\\
  University of Manitoba\\
  186 Dysart Road Winnipeg, MB, Canada R3T 2N2\\
  E-mail: \email{max.turgeon@umanitoba.ca}\\
  URL: \url{https://maxturgeon.ca/}\\~\\
      Jesse Islam\\
  McGill University\\
  1020 Pine Avenue West Montreal, QC, Canada H3A 1A2\\
  E-mail: \email{jesse.islam@mail.mcgill.ca}\\
  
      James Hanley\\
  McGill University\\
  1020 Pine Avenue West Montreal, QC, Canada H3A 1A2\\
  E-mail: \email{james.hanley@mcgill.ca}\\
  URL: \url{http://www.medicine.mcgill.ca/epidemiology/hanley/}\\~\\
      Olli Saarela\\
  University of Toronto\\
  Dalla Lana School of Public Health, 155 College Street, 6th floor,
  Toronto, Ontario M5T 3M7, Canada\\
  E-mail: \email{olli.saarela@utoronto.ca}\\
  URL: \url{http://individual.utoronto.ca/osaarela/}\\~\\
  }

% Pandoc header

\usepackage{amsmath} \usepackage{longtable} \usepackage{graphicx} \usepackage{tabularx} \usepackage{float} \usepackage{booktabs} \usepackage{makecell} \DeclareUnicodeCharacter{2500}{-}

\begin{document}

\hypertarget{introduction}{%
\section{Introduction}\label{introduction}}

Survival analysis has been greatly influenced over the last 50 years by
the partial likelihood approach of the Cox proportional hazard model
\citep{cox1972regression}. This approach provides a flexible way of
assessing the influence of covariates on the hazard function, without
the need to specify a parametric survival model. This flexibility comes
at the cost of decoupling the baseline hazard from the effect of the
covariates. To recover the whole survival curve---or the cumulative
incidence function (CIF)---we then need to separately estimate the
baseline hazard \citep{breslow1972discussion}. This in turn often leads
to stepwise estimates of the survival function.

From the perspective of clinicians and their patients, the most relevant
quantity is often the 5- or 10-year risk of having a certain event given
the patient's particular circumstances, and not the hazard ratio between
a treatment and control group. Therefore, to make sound clinical
decisions, it is important to accurately estimate the \emph{full} hazard
function, which can then be used to estimate the cumulative incidence
function (CIF). From this perspective, the CIF and the survival function
estimates are now smooth functions of time.

With the goal of fitting smooth-in-time hazard functions, Hanley \&
Miettinen \citeyearpar{hanley2009fitting} proposed a general framework
for estimating fully parametric hazard models via logistic regression.
Their approach provides users that are familiar with generalized linear
models with a natural way of fitting parametric survival models.
Moreover, their framework is very flexible: general functions of time
can be estimated (e.g.~using splines or general additive models), and
hence these models retain some of the flexibility of partial likelihood
approaches.

In this article, we present an \proglang{R} package that combines the
ideas of Hanley \& Miettinen into a simple interface. The purpose of the
\pkg{casebase} package is to provide practitioners with an easy-to-use
software tool to predict the risk (or cumulative incidence) of an event,
conditional on a particular patient's covariate profile. Our package
retains the flexibility of case-base sampling and the familiar interface
of the \code{glm} function.

In what follows, we first recall some theoretical details on case-base
sampling and its use for estimating parameteric hazard functions. We
then give a short review of existing \proglang{R} packages that
implement comparable features as \pkg{casebase}. Next, we provide some
details about the implementation of case-base sampling in our package,
and we give a brief survey of its main functions. This is followed by
four case studies that illustrate the flexibility and capabilities of
\pkg{casebase}. We show how the same framework can be used for competing
risk analyses, penalized estimation, and for studies with time-dependent
exposures. Finally, we end the article with a discussion of the results
and of future directions.

\hypertarget{theoretical-details}{%
\section{Theoretical details}\label{theoretical-details}}

As discussed in Hanley \& Miettinen \citeyearpar{hanley2009fitting}, the
key idea behind case-base sampling is to discretize the study base into
an infinite amount of \emph{person moments}. These person moments are
indexed by both an individual in the study and a time point, and
therefore each person moment has a covariate profile, an exposure status
and an outcome status attached to it. We note that there is only a
finite number of person moments associated with the event of interest
(what Hanley \& Miettinen call the \emph{case series}). The case-base
sampling refers to the sampling from the base of a representative finite
sample called the \emph{base series}.

To describe the theoretical foundations of case-base sampling, we use
the framework of counting processes. In what follows, we abuse notation
slightly and omit any mention of \(\sigma\)-algebras; the interested
reader can refer to Saarela \& Arjas \citeyearpar{saarela2015non} and
Saarela \citeyearpar{saarela2016case}. First, let
\(N_{i}(t) \in \{0, 1\}\) be counting processes corresponding to the
event of interest for individual \(i=1, \ldots,n\). For simplicity, we
will consider Type I censoring due to the end of follow-up at time
\(\tau\) (the general case of non-informative censoring is treated in
Saarela \citeyearpar{saarela2016case}). We assume a continuous time
model, which implies that the counting process jumps are less than or
equal to one. We are interested in modeling the hazard functions
\(\lambda_{i}(t)\) of the processes \(N_i(t)\), and which satisfy
\[\lambda_{i}(t) dt = E[dN_{i}(t)].\]

Next, we model the base series sampling mechanism using non-homogeneous
Poisson processes \(R_i(t) \in \{0, 1, 2, \ldots\}\), with the
person-moments where \(dR_i(t) = 1\) constituting the base series. The
process \(Q_{i}(t) = R_i(t) + N_{i}(t)\) then counts both the case and
base series person-moments contributed by individual \(i\). This process
is typically defined by the user via its hazard function \(\rho_i(t)\).
The process \(Q_{i}(t)\) is characterized by
\(E[dQ_{i}(t)] = \lambda_{i}(t)dt + \rho_i(t)dt\).

If the hazard function \(\lambda_{i}(t; \theta)\) is parametrized in
terms of \(\theta\), we could define an estimator \(\hat{\theta}\) by
maximization of the likelihood expression
\[L_0(\theta) = \prod_{i=1}^n \exp\left\{ -\int_0^\tau \lambda_i(t; \theta) dt \right\} \prod_{i=1}^{n} \prod_{t\in[0,\tau)} \lambda_{i}(t;\theta)^{dN_{i}(t)}\left(1 - \lambda_{i}(t;\theta)\right)^{1 - dN_{i}(t)},\]
where \(\prod_{t\in[0,u)}\) represents a product integral from \(0\) to
\(u\). However, the integral over time makes the computation and
maximisation of \(L_0(\theta)\) challenging.

Case-base sampling allows us to avoid this integral. By conditioning on
a sampled person-moment, we get individual contributions of the form
\[P(dN_{i}(t) \mid dQ_{i}(t) = 1) \stackrel{\theta}{\propto} \frac{\lambda_{i}(t; \theta)^{dN_{i}(t)}}{\rho_i(t) + \lambda_{i}(t;\theta)}.\]
Therefore, we can define an estimating equation for \(\theta\) as
follows:
\[L(\theta) = \prod_{i=1}^{n} \prod_{t\in[0,\tau)} \left(\frac{\lambda_{i}(t; \theta)^{dN_{i}(t)}}{\rho_i(t) + \lambda_{i}(t;\theta)}\right)^{dQ_i(t)}.\]
When a logarithmic link function is used for modeling the hazard
function, the above expression is of a logistic regression form with an
offset term \(\log(1/\rho_i(t))\). Note that the sampling units selected
in the case-base sampling mechanism are person-moments, rather than
individuals, and the parameters to be estimated are hazards or hazard
ratios rather than odds or odds ratios. Generally, an individual can
contribute more than one person-moment, and thus the terms in the
product integral are not independent. Nonetheless, Saarela
\citeyearpar{saarela2016case} showed that the logarithm of this
estimating function has mean zero at the true value \(\theta=\theta_0\),
and that the resulting estimator \(\hat{\theta}\) is asymptotically
normally distributed.

In Hanley \& Miettinen \citeyearpar{hanley2009fitting}, the authors
suggest sampling the base series \emph{uniformly} from the study base.
In terms of Poisson processes, this sampling strategy corresponds
essentially to a time-homogeneous Poisson process with hazard equal to
\(\rho_i(t) = b/B\), where \(b\) is the number of sampled observations
in the base series, and \(B\) is the total population-time for the study
base. More complex examples are also available; see for example Saarela
\& Arjas \citeyearpar{saarela2015non}, where the probabilities of the
sampling mechanism are proportional to the cardiovascular disease event
rate given by the Framingham score. With this sampling mechanism,
Saarela \& Arjas are able to increase the efficiency of their
estimators, when compared to uniform sampling.

Let \(g(t; X)\) be the linear predictor such that
\(\log(\lambda(t;X)) = g(t; X)\). Different functions of \(t\) lead to
different parametric hazard models. The simplest of these models is the
one-parameter exponential distribution which is obtained by taking the
hazard function to be constant over the range of \(t\):

\[ \log(\lambda(t; X)) = \beta_0 + \beta_1 X.\]

The instantaneous failure rate is independent of
\(t\).\footnote{The conditional chance of failure in a time interval of specified length is the same regardless of how long the individual has been in the study. This is also known as the \textit{memoryless property} \citep{kalbfleisch2011statistical}.}

The Gompertz hazard model is given by including a linear term for time:

\[ \log(\lambda(t; X)) = \beta_0 + \beta_1 t + \beta_2 X.\]

Use of \(\log(t)\) yields the Weibull hazard which allows for a power
dependence of the hazard on time \citep{kalbfleisch2011statistical}:

\[ \log(\lambda(t; X)) = \beta_0 + \beta_1 \log(t) + \beta_2 X. \]

For competing-risk analyses with \(J\) possible events, we can show that
each person-moment's contribution of the likelihood is of the form

\[\frac{\lambda_j(t)^{dN_j(t)}}{\rho(t) + \sum_{j=1}^J\lambda_j(t)},\]

where \(N_j(t)\) is the counting process associated with the event of
type \(j\) and \(\lambda_j(t)\) is the corresponding hazard function. As
may be expected, this functional form is similar to the terms appearing
in the likelihood function for multinomial
regression.\footnote{Specifically, it corresponds to the following parametrization: \begin{align*} \log\left(\frac{P(Y=j \mid X)}{P(Y = J \mid X)}\right) = X^T\beta_j, \qquad j = 1,\ldots, J-1.\end{align*}}

\hypertarget{existing-packages}{%
\section{Existing packages}\label{existing-packages}}

Survival analysis is an important branch of applied statistics and
epidemiology. Accordingly, there is a vast ecosystem of \code{R}
packages implementing different methods. In this section, we describe
how the functionalities of \pkg{casebase} compare to these packages.

At the time of writing, a cursory examination of CRAN's task view on
survival analysis reveals that there are over 250 packages related to
survival analysis \citeyearpar{survTaskView}. For the purposes of this
article, we restricted our description to packages that implement at
least one of the following features: parametric modeling,
non-proportional hazard models, competing risk analysis, penalized
estimation, and cumulative incidence curve estimation. By searching for
appropriate keywords in the \code{DESCRIPTION} file of these packages,
we found 60 relevant packages. These 60 packages were then manually
examined to determine which ones are comparable to \pkg{casebase}. In
particular, we excluded packages that were focused on a different set of
problems, such as frailty and multi-state models. The remaining 14
packages appear in Table \ref{tab:surv-pkgs}, along with some of the
functionalities they offer.

Several packages implement penalized estimation for the Cox model:
\pkg{glmnet} \citeyearpar{regpathcox}, \pkg{glmpath}
\citeyearpar{park_hastie}, \pkg{penalized} \citeyearpar{l1penal},
\pkg{RiskRegression} \citeyearpar{gerds_blanche}. Moreover, some
packages also include penalized estimation in the context of Cox models
with time-varying coefficients: elastic-net penalization with
\pkg{CoxRidge} \citeyearpar{perperoglou} and \pkg{rstpm2}
\citeyearpar{clements_liu}, while \pkg{survival}
\citeyearpar{survival-package} has an implementation for ridge.
\pkg{casebase} provides penalized estimation in the context of
parametric hazards. To our knowledge, \pkg{casebase} and \pkg{rsptm2}
are the only packages to offer this functionality.
\textcolor{red}{I can dive deeper to determine if this is a fair statement}

Parametric survival models are implemented in a handful of packages:
\pkg{CFC} \citeyearpar{mahani2015bayesian}, \pkg{flexsurv}
\citeyearpar{flexsurv}, \pkg{SmoothHazard} \citeyearpar{smoothHazard},
\pkg{rsptm2} \citeyearpar{clements_liu}, \pkg{mets}
\citeyearpar{scheike2014estimating}, and \pkg{survival}. The types of
models they allow vary for each package. For example, \pkg{SmoothHazard}
is limited to Weibull distributions \citeyearpar{smoothHazard}, whereas
both \pkg{flexsurv} and \pkg{survival} allow users to supply any
distribution of their choice. Also, \pkg{flexsurv}, \pkg{smoothhazard},
\pkg{mets} and \pkg{rstpm2} also have the ability to model the effect of
time using splines, which allows flexible modeling of the hazard
function. Moreover, \pkg{flexsurv} has the ability to estimate both
scale and shape parameters for a variety of parametric families (see
Table \ref{tab:dists}). As discussed above, \pkg{casebase} can model any
parametric family whose log-hazard can be expressed as a linear model of
covariates (including time). Therefore, our package allows the user to
model the effect of time using splines, and through interaction terms
involving covariates and time, it also allows user to fit time-varying
coefficient models. However, we do not explicitly model any shape
parameter, unlike \pkg{flexsurv}.

Several \proglang{R} packages implement methodologies for competing risk
analysis; for a different perspective, see Mahani \& Sharabiani
\citeyearpar{mahani2015bayesian}. The package \pkg{cmprsk} provides
methods for cause-specific subdistributions, such as in the Fine-Gray
model \citeyearpar{fine1999proportional}. On the other hand, the package
\pkg{CFC} estimates cause-specific cumulative incidence functions from
unadjusted, non-parametric survival functions. Our package
\pkg{casebase} also provides functionalities for competing risk analysis
by estimating the cause-specific hazards, and from these quantities, we
can derive the cause-specific cumulative incidence functions.

Finally, several packages include functions to estimate the cumulative
incidence function. The corresponding methods generally fall into two
categories: transformation of the estimated hazard function, and
semi-parametric estimation of the baseline hazard. The first category
broadly corresponds to parametric survival models, where the full hazard
is explictly modeled. Using this estimate, the survival function and the
cumulative incidence function can be obtained using their functional
relationships (see Equations \ref{eqn:surv} and \ref{eqn:CI} below).
Packages including this functionality include \pkg{CFC}, \pkg{Flexsurv},
\pkg{mets}, and \pkg{survival}. Our package \pkg{casebase} also follows
this approach for both single-event and competing-event analyses. The
second category outlined above broadly corresponds to Cox models. These
models do not model the full hazard function, and therefore the baseline
hazard needs to be estimated separately in order to estimate the
survival function. This is achieved using semi-parametric estimators
(e.g.~Breslow's estimator). Packages that implement this approach
include \pkg{RiskRegression}, \pkg{rstpm2}, and \pkg{survival}. As
mentioned in the introduction, a key distinguishing factor between these
two approaches is that the first category leads to smooth estimates of
the cumulative incidence function, whereas the second category produces
estimates in the form of step-wise functions. This was one of the main
motivations for introducting case-base sampling in survival analysis.

\begin{table}[ht]
\centering
\resizebox{\textwidth}{!}{%
\begin{tabular}{ccccccccc}
\hline
\textbf{Package}        & \vtop{\hbox{\strut \textbf{Competing}}\hbox{\strut \textbf{risks}}}  & \textbf{Non-proportional} & \textbf{Penalization} & \textbf{Splines} & \textbf{Parametric} & \textbf{Semi-parametric} & \vtop{\hbox{\strut \textbf{Interval/left}}\hbox{\strut \textbf{censoring}}} & \vtop{\hbox{\strut \textbf{Absolute}}\hbox{\strut \textbf{risk}}}           \\
\hline
\textbf{casebase} \cite{hanley2009fitting}       & x                        & x                         & x                     & x                & x                   &                          &                                  & x                                \\
\textbf{CFC} \cite{mahani2015bayesian}            & x                        & x                         &                       &                  & x                   &                          &                                  & x                                \\
textbf{cmprsk} \cite{gray_2019}            & x                        &                           &                       &                  &                     &  x                        &                                  & x                                \\
\textbf{coxRidge} \cite{perperoglou}       &                          & x                         & x                     &                  &                     & x                        &                                  &                                  \\
\textbf{crrp} \cite{fu}           & x                        &                           & x                     &                  &                     &                          &                                  &                                  \\
\textbf{fastcox} \cite{yi_zou}        &                          &                           & x                     &                  &                     & x                        &                                  &                                  \\
\textbf{Flexrsurv} \cite{flexrsurv}      &                          & x                         &                       & x                & x                   &                          &                                  & x               \\
\textbf{Flexsurv} \cite{flexsurv}       & x                        & x                         &                       & x                & x                   &                          &                                  & x               \\
\textbf{glmnet} \cite{regpathcox}         &                          &                           & x                     &                  &                     & x                        &                                  &                                  \\
\textbf{glmpath} \cite{park_hastie}       &                          &                           & x                     &                  &                     & x                        &                                  &                                  \\
\textbf{mets} \cite{scheike2014estimating}           & x                        &                           &                       & x                &                     & x                        &                                  & x                                \\
\textbf{penalized} \cite{goeman_meijer2019}      &                          &                           & x                     &                  &                     & x                        &                                  &                                  \\
\textbf{RiskRegression} \cite{gerds_blanche} &                          &                           & x                     &                  &                     & x                        &                                  & x                                \\
\textbf{Rstpm2} \cite{clements_liu}         &                          & x                         & x                     & x                & x                   & x                        & x                                & x                         \\
\textbf{SmoothHazard} \cite{smoothHazard}   &                          & x                         &                       & x                & x                   &                          & x                            &                                      \\
\textbf{Survival} \cite{survival-package}       & x                        & x                         &                       &                  & x                   & x                        & x                                & x                               
\end{tabular}%
}
\caption{Different features of interest in various survival packages.}
\label{tab:surv-pkgs}
\end{table}

\begin{table}
  \begin{tabular}{p{1.8in}lll}
\hline
    &  Parameters &  Density \proglang{R} function & \code{dist}\\
    & {\footnotesize{(location in \code{\color{red}{red}})}} & & \\
\hline
    Exponential & \code{\color{red}{rate}}             & \code{dexp}   & \code{"exp"} \\
    Weibull (accelerated failure time)     & \code{shape, {\color{red}{scale}}}     & \code{dweibull} & \code{"weibull"} \\
    Weibull (proportional hazards)     & \code{shape, {\color{red}{scale}}}     & \code{dweibullPH} & \code{"weibullPH"} \\
    Gamma       & \code{shape, \color{red}{rate}}      & \code{dgamma} & \code{"gamma"}\\
    Log-normal  & \code{{\color{red}{meanlog}}, sdlog}   & \code{dlnorm} & \code{"lnorm"}\\
    Gompertz    & \code{shape, {\color{red}{rate}}}      & \code{dgompertz} & \code{"gompertz"} \\
    Log-logistic & \code{shape, {\color{red}{scale}}}   & \code{dllogis} & \code{"llogis"}\\
    Generalized gamma (Prentice 1975)   & \code{{\color{red}{mu}}, sigma, Q} & \code{dgengamma} & \code{"gengamma"} \\
    Generalized gamma (Stacy 1962)& \code{shape, {\color{red}{scale}}, k} & \code{dgengamma.orig} & \code{"gengamma.orig"} \\
    Generalized F     (stable)    & \code{{\color{red}{mu}}, sigma, Q, P} & \code{dgenf} & \code{"genf"} \\
    Generalized F     (original)  & \code{{\color{red}{mu}}, sigma, s1, s2} & \code{dgenf.orig} & \code{"genf.orig"} \\
\hline
  \end{tabular}
  \caption{Built-in parametric survival distributions in \pkg{flexsurv}.}
  \label{tab:dists}
\end{table}

\hypertarget{implementation-details}{%
\section{Implementation details}\label{implementation-details}}

The functions in the casebase package can be divided into two
categories: 1) data visualization, in the form of population-time plots;
and 2) parametric modeling. We explicitly aimed at being compatible with
both \code{data.frame}s and \code{data.table}s. This is evident in some
of the coding choices we made, and it is also reflected in our unit
tests.

\hypertarget{population-time-plots}{%
\subsection{Population-time plots}\label{population-time-plots}}

\hypertarget{parametric-modeling}{%
\subsection{Parametric modeling}\label{parametric-modeling}}

The parametric modeling step was separated into three parts:

\begin{enumerate}
\def\labelenumi{\arabic{enumi}.}
\tightlist
\item
  case-base sampling;
\item
  estimation of the smooth hazard function;
\item
  calculation of the risk function.
\end{enumerate}

By separating the sampling and estimation functions, we allowed the
possibility of users implementing more complex sampling scheme, as
described in Saarela \citeyearpar{saarela2016case}.

The sampling scheme selected for \code{sampleCaseBase} was described in
Hanley and Miettinen \citeyearpar{hanley2009fitting}: we first sample
along the ``person'' axis, proportional to each individual's total
follow-up time, and then we sample a moment uniformly over their
follow-up time. This sampling scheme is equivalent to the following
picture: imagine representing the total follow-up time of all
individuals in the study along a single dimension, where the follow-up
time of the next individual would start exactly when the follow-up time
of the previous individual ends. Then the base series could be sampled
uniformly from this one-dimensional representation of the overall
follow-up time. In any case, the output is a dataset of the same class
as the input, where each row corresponds to a person-moment. The
covariate profile for each such person-moment is retained, and an offset
term is added to the dataset. This output could then be used to fit a
smooth hazard function, or for visualization of the base series.

The fitting function \code{fitSmoothHazard} starts by looking at the
class of the dataset: if it was generated from \code{sampleCaseBase}, it
automatically inherited the class \code{cbData}. If the dataset supplied
to \code{fitSmoothHazard} does not inherit from \code{cbData}, then the
fitting function starts by calling \code{sampleCaseBase} to generate the
base series. In other words, the occasional user can bypass
\code{sampleCaseBase} altogether and only worry about the fitting
function \code{fitSmoothHazard}.

The fitting function retains the familiar formula interface of
\code{glm}. The left-hand side of the formula should be the name of the
column corresponding to the event type. The right-hand side can be any
combination of the covariates, along with an explicit functional form
for the time variable. Note that non-proportional hazard models can be
achieved at this stage by adding an interaction term involving time. The
offset term does not need to be specified by the user, as it is
automatically added to the formula.

To fit the hazard function, we provide several approaches that are
available via the \code{family} parameter. These approaches are:

\begin{itemize}
\tightlist
\item
  \code{glm}: This is the familiar logistic regression.
\item
  \code{glmnet}: This option allows for variable selection using Lasso
  or elastic-net. This functionality is provided through the
  \code{glmnet} package \citep{friedman2010jss}.
\item
  \code{gam}: This option provides support for \emph{Generalized
  Additive Models} via the \code{gam} package
  \citep{hastie1987generalized}.
\item
  \code{gbm}: This option provides support for \emph{Gradient Boosted
  Trees} via the \code{gbm} package. This feature is still experimental.
\end{itemize}

In the case of multiple events, the hazard is fitted via multinomial
regression as performed by the \pkg{VGAM} package. This package was
selected for its ability to fit multinomial regression models with an
offset.

Once a model-fit object has been returned by \code{fitSmoothHazard}, all
the familiar summary and diagnostic functions are available:
\code{print}, \code{summary}, \code{predict}, \code{plot}, etc. Our
package provides one more functionality: it computes risk functions from
the model fit. For the case of a single event, it uses the familiar
identity \begin{equation}\label{eqn:surv}
S(t) = \exp\left(-\int_0^t h(u;X) du\right).
\end{equation} The integral is computed using either the
\code{stats::integrate} function or Monte-Carlo integration. The risk
function (or cumulative incidence function) is then defined as
\begin{equation}\label{eqn:CI}
CI(t) = 1 - S(t).
\end{equation}

For the case of a competing-event analysis, the event-specific risk is
computed using the following procedure: first, we compute the overall
survival function (i.e.~for all event types):

\[ S(t) = \exp\left(-\int_0^t H(u;X) du\right),\qquad H(t;X) = \sum_{j=1}^J h_j(t;X).\]
From this, we can derive the event-specific subdensities:

\[ f_j(t) = h_j(t)S(t).\]

Finally, by integrating these subdensities, we obtain the event-specific
cumulative incidence functions:

\[ CI_j(t) = \int_0^t f_j(u)du.\]

We created \code{absoluteRisk} as an \code{S3} generic, with methods for
the different types of outputs of \code{fitSmoothHazard}. The method
dispatch system of \proglang{R} then takes care of matching the correct
output to the correct methodology for calculating the cumulative
incidence function, without the user's intervention.

In the following sections, we illustrate these functionalities in the
context of three case studies.

\hypertarget{case-study-1european-randomized-study-of-prostate-cancer-screening}{%
\section{Case study 1--European Randomized Study of Prostate Cancer
Screening}\label{case-study-1european-randomized-study-of-prostate-cancer-screening}}

For the first case study, we will use of the European Randomized Study
of Prostate Cancer Screening data. This dataset is available through the
\pkg{casebase} package:

\begin{CodeChunk}

\begin{CodeInput}
R> data(ERSPC)
R> ERSPC$ScrArm <- factor(ERSPC$ScrArm, 
R>                        levels = c(0, 1), 
R>                        labels = c("Control group", "Screening group"))
\end{CodeInput}
\end{CodeChunk}

The results of this study were published by
\citep{schroder2009screening}, and the dataset itself was obtained using
the approach described in \citep{liu2014recovering}.

Population time plots can be extremely informative graphical displays of
survival data. They should be the first step in an exploratory data
analysis. We facilitate this task in the \pkg{casebase} package using
the \code{popTime} function. We first create the necessary dataset for
producing the population time plots, and we can generate the plot by
using the corresponding \code{plot} method:

\begin{CodeChunk}

\begin{CodeInput}
R> pt_object <- casebase::popTime(ERSPC, event = "DeadOfPrCa")
R> plot(pt_object)
\end{CodeInput}
\end{CodeChunk}

We can also create exposure stratified plots by specifying the
\code{exposure} argument in the \code{popTime} function:

\begin{CodeChunk}

\begin{CodeInput}
R> pt_object_strat <- casebase::popTime(ERSPC, 
R>                                      event = "DeadOfPrCa", 
R>                                      exposure = "ScrArm")
R> plot(pt_object_strat)
\end{CodeInput}
\end{CodeChunk}

We can also plot them side-by-side using the \code{ncol} argument:

\begin{CodeChunk}

\begin{CodeInput}
R> plot(pt_object_strat, ncol = 2)
\end{CodeInput}
\end{CodeChunk}

ADD PARAGRAPH ABOUT WHAT WE CAN CONCLUDE FROM THESE GRAPHS.

Next, we investigate the differences between the control and the
screening arms. A common choice for this type of analysis is to use a
Cox regression model and estimate the hazard ratio for the screening
group (relative to the control group). In \proglang{R}, it can be done
as follows:

\begin{CodeChunk}

\begin{CodeInput}
R> cox_model <- survival::coxph(Surv(Follow.Up.Time, DeadOfPrCa) ~ ScrArm, 
R>                              data = ERSPC)
R> summary(cox_model)
\end{CodeInput}
\end{CodeChunk}

We can also plot the cumulative incidence function (CIF) for each group.
However, this involves estimating the baseline hazard, which Cox
regression treated as a nuisance parameter.

\begin{CodeChunk}

\begin{CodeInput}
R> new_data <- data.frame(ScrArm = c("Control group", "Screening group"),
R>                        ignore = 99)
R> 
R> plot(survfit(cox_model, newdata = new_data),
R>      xlab = "Years since Randomization", 
R>      ylab = "Cumulative Incidence (%)", 
R>      fun = "event",
R>      xlim = c(0,15), conf.int = FALSE, col = c("red","blue"), 
R>      main = sprintf("Estimated Cumulative Incidence (risk) of Death from Prostate 
R>                     Cancer Screening group Hazard Ratio: %.2g (%.2g, %.2g)",
R>                     exp(coef(cox_model)), 
R>                     exp(confint(cox_model))[1], 
R>                     exp(confint(cox_model))[2]),
R>      ylim = c(0, 2), yscale = 100)
R> legend("topleft", 
R>        legend = c("Control group", "Screening group"), 
R>        col = c("red","blue"),
R>        lty = c(1, 1), 
R>        bg = "gray90")
\end{CodeInput}
\end{CodeChunk}

Next, we show how case-base sampling can be used to carry out a similar
analysis. We fit several models that differ in how we choose to model
time.

First, we will fit an exponential model. Recall that this corresponds to
excluding time from the linear predictor.

\begin{CodeChunk}

\begin{CodeInput}
R> casebase_exponential <- casebase::fitSmoothHazard(DeadOfPrCa ~ ScrArm, 
R>                                                   data = ERSPC, 
R>                                                   ratio = 100)
R> 
R> summary(casebase_exponential)
R> exp(coef(casebase_exponential)[2])
R> exp(confint(casebase_exponential)[2,])
\end{CodeInput}
\end{CodeChunk}

We can then use the \code{absoluteRisk} function to get an estimate of
the cumulative incidence curve for a specific covariate profile. In the
plot below, we overlay the estimated CIF from the exponential model on
the Cox model CIF:

\begin{CodeChunk}

\begin{CodeInput}
R> smooth_risk_exp <- casebase::absoluteRisk(object = casebase_exponential, 
R>                                           time = seq(0,15,0.1), 
R>                                           newdata = new_data)
R> 
R> plot(survfit(cox_model, newdata = new_data),
R>      xlab = "Years since Randomization", 
R>      ylab = "Cumulative Incidence (%)", 
R>      fun = "event",
R>      xlim = c(0,15), conf.int = FALSE, col = c("red","blue"), 
R>      main = sprintf("Estimated Cumulative Incidence (risk) of Death from Prostate 
R>                     Cancer Screening group Hazard Ratio: %.2g (%.2g, %.2g)",
R>                     exp(coef(cox_model)), 
R>                     exp(confint(cox_model))[1], 
R>                     exp(confint(cox_model))[2]),
R>      ylim = c(0, 2), yscale = 100)
R> lines(smooth_risk_exp[,1], smooth_risk_exp[,2], col = "red", lty = 2)
R> lines(smooth_risk_exp[,1], smooth_risk_exp[,3], col = "blue", lty = 2)
R> 
R> 
R> legend("topleft", 
R>        legend = c("Control group (Cox)","Control group (Casebase)",
R>                   "Screening group (Cox)", "Screening group (Casebase)"), 
R>        col = c("red","red", "blue","blue"),
R>        lty = c(1, 2, 1, 2), 
R>        bg = "gray90")
\end{CodeInput}
\end{CodeChunk}

As we can see, the exponential model gives us an estimate of the hazard
ratio that is similar to the one obtained using Cox regression. However,
the CIF estimates are quite different. Based on what we observed in the
population time plot, where more events are observed later on in time,
we do not expect a constant hazard would be a good description for this
data. In other words, this poor fit for the exponential hazard is
expected. A constant hazard model would overestimate the cumulative
incidence earlier on in time, and underestimate it later on; this is
what we see on the cumulative incidence plot. This example demonstrates
the benefits of population time plots as an exploratory analysis tool.

For the next model, we include time as a linear term. Recall that this
corresponds to a Weibull model.

\begin{CodeChunk}

\begin{CodeInput}
R> casebase_time <- fitSmoothHazard(DeadOfPrCa ~ Follow.Up.Time + ScrArm, 
R>                                  data = ERSPC, 
R>                                  ratio = 100)
R> 
R> summary(casebase_time)
R> exp(coef(casebase_time))
R> exp(confint(casebase_time))
\end{CodeInput}
\end{CodeChunk}

Again, the estimate of the hazard ratio is similar to that obtained from
Cox regression. We then look at the estimate of the CIF:

\begin{CodeChunk}

\begin{CodeInput}
R> smooth_risk_time <- casebase::absoluteRisk(object = casebase_time, 
R>                                           time = seq(0,15,0.1), 
R>                                           newdata = new_data)
R> 
R> plot(survfit(cox_model, newdata = new_data),
R>      xlab = "Years since Randomization", 
R>      ylab = "Cumulative Incidence (%)", 
R>      fun = "event",
R>      xlim = c(0,15), conf.int = FALSE, col = c("red","blue"), 
R>      main = sprintf("Estimated Cumulative Incidence (risk) of Death from Prostate 
R>                     Cancer Screening group Hazard Ratio: %.2g (%.2g, %.2g)",
R>                     exp(coef(cox_model)), 
R>                     exp(confint(cox_model))[1], 
R>                     exp(confint(cox_model))[2]),
R>      ylim = c(0, 2), yscale = 100)
R> lines(smooth_risk_time[,1], smooth_risk_time[,2], col = "red", lty = 2)
R> lines(smooth_risk_time[,1], smooth_risk_time[,3], col = "blue", lty = 2)
R> 
R> legend("topleft", 
R>        legend = c("Control group (Cox)","Control group (Casebase)",
R>                   "Screening group (Cox)", "Screening group (Casebase)"), 
R>        col = c("red","red", "blue","blue"),
R>        lty = c(1, 2, 1, 2), 
R>        bg = "gray90")
\end{CodeInput}
\end{CodeChunk}

We see that the Weibull model leads to a better fit of the CIF than the
Exponential model, when compared with the semi-parametric approach.

Finally, we can model the hazard as a smooth function of time using the
\code{splines} package:

\begin{CodeChunk}

\begin{CodeInput}
R> casebase_splines <- fitSmoothHazard(DeadOfPrCa ~ bs(Follow.Up.Time) + ScrArm, 
R>                                     data = ERSPC, 
R>                                     ratio = 100)
R> 
R> summary(casebase_splines)
R> exp(coef(casebase_splines))
R> exp(confint(casebase_splines))
\end{CodeInput}
\end{CodeChunk}

Once again, we can see that the estimate of the hazard ratio is similar
to that from the Cox regression. We then look at the estimate of the
CIF:

\begin{CodeChunk}

\begin{CodeInput}
R> smooth_risk_splines <- absoluteRisk(object = casebase_splines, 
R>                                     time = seq(0,15,0.1), 
R>                                     newdata = new_data)
R> 
R> plot(survfit(cox_model, newdata = new_data),
R>      xlab = "Years since Randomization", 
R>      ylab = "Cumulative Incidence (%)", 
R>      fun = "event",
R>      xlim = c(0,15), conf.int = FALSE, col = c("red","blue"), 
R>      main = sprintf("Estimated Cumulative Incidence (risk) of Death from Prostate 
R>                     Cancer Screening group Hazard Ratio: %.2g (%.2g, %.2g)",
R>                     exp(coef(cox_model)), 
R>                     exp(confint(cox_model))[1], 
R>                     exp(confint(cox_model))[2]),
R>      ylim = c(0, 2), yscale = 100)
R> lines(smooth_risk_splines[,1], smooth_risk_splines[,2], col = "red", lty = 2)
R> lines(smooth_risk_splines[,1], smooth_risk_splines[,3], col = "blue", lty = 2)
R> 
R> legend("topleft", 
R>        legend = c("Control group (Cox)","Control group (Casebase)",
R>                   "Screening group (Cox)", "Screening group (Casebase)"), 
R>        col = c("red","red", "blue","blue"),
R>        lty = c(1, 2, 1, 2), 
R>        bg = "gray90")
\end{CodeInput}
\end{CodeChunk}

Qualitatively, the combination of splines and case-base sampling
produces the parametric model that most closely resembles the Cox model.
In the following table, we see that the confidence intervals are also
similar across all four models. In other words, this reinforces the idea
that, under proportional hazards, we do not need to model the full
hazard to obtain reliable estimates of the hazard ratio. Of course,
different parametric models for the hazards give rise to qualitatively
different estimates for the CIF.

As noted above, the usual asymptotic results hold for likelihood ratio
tests built using case-base sampling models. Therefore, we can easily
test the null hypothesis that the exponential model is just as good as
the larger (in terms of number of parameters) splines model.

\begin{CodeChunk}

\begin{CodeInput}
R> anova(casebase_exponential, casebase_splines, test = "LRT")
\end{CodeInput}
\end{CodeChunk}

As expected, we see that splines model provides a better fit. Similarly,
we can compare the AIC for all three case-base models:

\begin{CodeChunk}

\begin{CodeInput}
R> c("Exp." = AIC(casebase_exponential),
R>   "Weib." = AIC(casebase_time),
R>   "Splines" = AIC(casebase_splines))
\end{CodeInput}
\end{CodeChunk}

Once again, we have evidence that the splines model provides the best
fit.

With this first case study, we saw how population-time plots can provide
a useful summary of the incidence for our event of interests. In
particular, we saw how it can provide evidence against an exponential
hazard model. We then explored how different parametric models can
provide different estimates of the CIF, even though they all give
similar estimates for the hazard ratios.

\hypertarget{case-study-2bone-marrow-transplant}{%
\section{Case study 2--Bone-marrow
transplant}\label{case-study-2bone-marrow-transplant}}

In the next case study, we show how case-base sampling can be used in
the context of a competing risk analysis. For illustrative purposes, we
will use the same data that was used in Scrucca \emph{et al}
\citeyearpar{scrucca2010regression}. The data was downloaded from the
first author's website, and it is also available as part of the
\pkg{casebase} package.

\begin{CodeChunk}

\begin{CodeInput}
R> 
R> data(bmtcrr)
\end{CodeInput}
\end{CodeChunk}

The data contains information on 177 patients who received a stem-cell
transplant for acute leukemia. The event of interest is relapse, but
other competing causes (e.g.~transplant-related death) were also
recorded. Several covariates were captured at baseline: sex, disease
type (acute lymphoblastic or myeloblastic leukemia, abbreviated as ALL
and AML, respectively), disease phase at transplant (Relapse, CR1, CR2,
CR3), source of stem cells (bone marrow and peripheral blood, coded as
BM+PB, or only peripheral blood, coded as PB), and age. A summary of
these baseline characteristics appear in Table \ref{tab:table1bmtcrr}.
We note that the statistical summaries were generated differently for
different variable types: for continuous variables, we gave the range,
followed by the mean and standard deviation; for categorical variables,
we gave the counts for each category.

First, we can look at a population-time plot to visualize the incidence
density of both relapse and the competing events. In Figure
\ref{fig:compPop}, failure times are highlighted on the plot using red
dots for the event of interest (panel A) and blue dots for competing
events (panel B). In both panels, we see evidence of a non-constant
hazard function: the density of points is larger at the beginning of
follow-up than at the end.

Our main objective is to compute the absolute risk of relapse for a
given set of covariates. We start by fitting a smooth hazard to the data
using a linear term for time:

\begin{CodeChunk}

\begin{CodeInput}
R> model_cb <- fitSmoothHazard(
R>     Status ~ ftime + Sex + D + Phase + Source + Age, 
R>     data = bmtcrr, 
R>     ratio = 100, 
R>     time = "ftime")
\end{CodeInput}
\end{CodeChunk}

We will compare our hazard ratio estimates to that obtained from a Cox
regression. To do so, we need to treat the competing event as censoring.

\begin{CodeChunk}

\begin{CodeInput}
R> library(survival)
R> # Treat competing event as censoring
R> model_cox <- coxph(Surv(ftime, Status == 1) ~ Sex + D + Phase + Source + Age,
R>                    data = bmtcrr)
\end{CodeInput}
\end{CodeChunk}

From the fit object, we can extract both the hazard ratios and their
corresponding confidence intervals. These quantities appear in Table
\ref{tab:bmtcrr-cis}. As we can see, the only significant hazard ratio
identified by case-base sampling is the one associated with the phase of
the disease at transplant. More precisely, being in relapse at
transplant is associated with a hazard ratio of 3.89 when compared to
CR1.

Given the estimate of the hazard function obtained using case-base
sampling, we can compute the absolute risk curve for a fixed covariate
profile. We perform this computation for a 35 year old woman who
received a stem-cell transplant from peripheral blood at relapse. We
compared the absolute risk curve for such a woman with acute
lymphoblastic leukemia with that for a similar woman with acute
myeloblastic leukemia. We will estimate the curve from 0 to 60 months.

\begin{CodeChunk}

\begin{CodeInput}
R> # Pick 100 equidistant points between 0 and 60 months
R> time_points <- seq(0, 60, length.out = 50)
R> 
R> # Data.frame containing risk profile
R> newdata <- data.frame("Sex" = factor(c("F", "F"), 
R>                                      levels = levels(bmtcrr[,"Sex"])),
R>                       "D" = c("ALL", "AML"), # Both diseases
R>                       "Phase" = factor(c("Relapse", "Relapse"), 
R>                                        levels = levels(bmtcrr[,"Phase"])),
R>                       "Age" = c(35, 35),
R>                       "Source" = factor(c("PB", "PB"), 
R>                                         levels = levels(bmtcrr[,"Source"])))
R> 
R> # Estimate absolute risk curve
R> risk_cb <- absoluteRisk(object = model_cb, time = time_points,
R>                         method = "numerical", newdata = newdata)
\end{CodeInput}
\end{CodeChunk}

We will compare our estimates to that obtained from a corresponding
Fine-Gray model \citeyearpar{fine1999proportional}. The Fine-Gray model
is a semiparametric model for the cause-specific \emph{subdistribution},
i.e.~the function \(f_k(t)\) such that
\[CI_k(t) =1 - \exp\left( - \int_0^t f_k(u) \textrm{d}u \right),\] where
\(CI_k(t)\) is the cause-specific cumulative indicence. The Fine-Gray
model allows to directly assess the effect of a covariate on the
subdistribution, as opposed to the hazard. For the computation, we will
use the \pkg{timereg} package \citep{timereg}:

\begin{CodeChunk}

\begin{CodeInput}
R> library(timereg)
R> model_fg <- comp.risk(Event(ftime, Status) ~ const(Sex) + const(D) + 
R>                           const(Phase) + const(Source) + const(Age), 
R>                       data = bmtcrr, cause = 1, model = "fg")
R> 
R> # Estimate absolute risk curve
R> risk_fg <- predict(model_fg, newdata, times = time_points)
\end{CodeInput}
\end{CodeChunk}

Figure \ref{fig:compAbsrisk} shows the absolute risk curves for both
case-base sampling and the Fine-Gray model. As we can see, the two
approaches agree quite well for acute myeloblastic leukemia; however,
there seems to be a difference of about 5\% between the two curves for
acute lymphoblastic leukemia. This difference does not appear to be
significant: the curve from case-base sampling is contained within a
95\% confidence band around the Fine-Gray absolute risk curve (figure
not shown).

\hypertarget{case-study-3support-data}{%
\section{Case study 3--SUPPORT Data}\label{case-study-3support-data}}

In the first two case studies, we described the basic functionalities of
the \pkg{casebase} package: creating population-time plots, fitting
parametric models for hazard functions, and estimating the corresponding
cumulative incidence curves. In the next two case studies, we explore
more advanced features.

For the third case study, we will look at regularized estimation for the
hazard function. This can be performed using a combination of case-base
sampling and regularized logistic regression. To demonstrate this
capability, we examine the Study to Understand Prognoses Preferences
Outcomes and Risks of Treatment (SUPPORT) dataset
\citep{knaus1995support}. The SUPPORT dataset tracks death for
individuals who are considered seriously ill within a hospital.
Imputation was conducted using the \pkg{mice} package in \proglang{R}
with default settings \citep{mice}. Before imputation, there were a
total of 1000 individuals and 35 variables. After imputation, we have
690 individuals. We lose individuals due to coliniarity. If a pair of
covariates are deemed colinear to a certain degree, they are excluded
from any imputations. Here, only edu income avtisst* race ph* glucose*
adlp* were imputed. Of these, 66.09\% experienced the event of interest.
A description of each variable can be found in Table \ref{tab:support1}
and a breakdown of each categorical variable in Table
\ref{tab:support2}. The original data was obtained from the Department
of Biostatistics at Vanderbilt University. The imputed dataset is
available as part of the \pkg{casebase} package.

\begin{table}[ht]
\centering
\begin{tabular}{ccccc}
\hline
\textbf{Name} & \textbf{Labels}                          & \textbf{Levels} & \textbf{Storage} & \textbf{NAs} \\
\hline
age           & Age                                      &                 & double           & 0            \\
death         & Death at any time up to NDI date:31DEC94 &                 & double           & 0            \\
sex           &                                          & 2               & integer          & 0            \\
hospdead      & Death in Hospital                        &                 & double           & 0            \\
slos          & Days from Study Entry to Discharge       &                 & double           & 0            \\
d.time        & Days of Follow-Up                        &                 & double           & 0            \\
dzgroup       &                                          & 8               & integer          & 0            \\
dzclass       &                                          & 4               & integer          & 0            \\
num.co        & number of comorbidities                  &                 & double           & 0            \\
edu           & Years of Education                       &                 & double           & 202          \\
income        &                                          & 4               & integer          & 349          \\
scoma         & SUPPORT Coma Score based on Glasgow D3   &                 & double           & 0            \\
charges       & Hospital Charges                         &                 & double           & 25           \\
totcst        & Total RCC cost                           &                 & double           & 105          \\
totmcst       & Total micro-cost                         &                 & double           & 372          \\
avtisst       & Average TISS, Days 3-25                  &                 & double           & 6            \\
race          &                                          & 5               & integer          & 5            \\
meanbp        & Mean Arterial Blood Pressure Day 3       &                 & double           & 0            \\
wblc          & White Blood Cell Count Day 3             &                 & double           & 24           \\
hrt           & Heart Rate Day 3                         &                 & double           & 0            \\
resp          & Respiration Rate Day 3                   &                 & double           & 0            \\
temp          & Temperature (Celsius) Day 3              &                 & double           & 0            \\
pafi          & PaO2/(.01*FiO2) Day 3                    &                 & double           & 253          \\
alb           & Serum Albumin Day 3                      &                 & double           & 378          \\
bili          & Bilirubin Day 3                          &                 & double           & 297          \\
crea          & Serum creatinine Day 3                   &                 & double           & 3            \\
sod           & Serum sodium Day 3                       &                 & double           & 0            \\
ph            & Serum pH (arterial) Day 3                &                 & double           & 250          \\
glucose       & Glucose Day 3                            &                 & double           & 470          \\
bun           & BUN Day 3                                &                 & double           & 455          \\
urine         & Urine Output Day 3                       &                 & double           & 517          \\
adlp          & ADL Patient Day 3                        &                 & double           & 634          \\
adls          & ADL Surrogate Day 3                      &                 & double           & 310          \\
sfdm2         &                                          & 5               & integer          & 159          \\
adlsc         & Imputed ADL Calibrated to Surrogate      &                 & double           & 0           
\end{tabular}
\caption{A description of each variable in the SUPPORT dataset.}
\label{tab:support1}
\end{table}

\begin{table}[ht]
\centering
\begin{tabular}{cc}
\hline
\textbf{Variable} & \textbf{Levels}                      \\
\hline
sex      & female                                        \\
         & male                                          \\
dzgroup  & ARF/MOSF w/Sepsis                             \\
         & COPD                                          \\
         & CHF                                           \\
         & Cirrhosis                                     \\
         & Coma                                          \\
         & Colon Cancer                                  \\
         & Lung Cancer                                   \\
         & MOSF w/Malig                                  \\
dzclass  & ARF/MOSF                                      \\
         & COPD/CHF/Cirrhosis                            \\
         & Coma                                          \\
         & Cancer                                        \\
income   & under \$11k                                   \\
         & $11-$25k                                      \\
         & $25-$50k                                      \\
         & \textgreater{}\$50k                           \\
race     & white                                         \\
         & black                                         \\
         & asian                                         \\
         & other                                         \\
         & hispanic                                      \\
sfdm2    & no(M2 and SIP pres)                           \\
         & adl\textgreater{}=4 (\textgreater{}=5 if sur) \\
         & SIP\textgreater{}=30                          \\
         & Coma or Intub                                 \\
         & \textless{}2 mo. follow-up                   
\end{tabular}
\caption{A description of each level within each categorical variable in the dataset.}
\label{tab:support2}
\end{table}

As before, the first step is to visualize the incidence density. The
population-time plot for the SUPPORT data appears in Figure
\ref{fig:support-poptime}.

\begin{CodeChunk}

\begin{CodeInput}
R> data(support)
R> support_popTime <- popTime(support,
R>                            time = "d.time", event = "death",percentile_number = 0.31)
R> plot(support_popTime)
\end{CodeInput}
\end{CodeChunk}

Before we proceed with the analysis, we first remove hospital death as a
covariate, as this is directly informative of death. All covariates will
be used to estimate the hazard function. The hazard function will be
used to estimate the cumulative incidence of each individual in the
study. Then, the average cumulative incidience curve for all
observations will be used for our comparison.

Our main objective is to compute the absolute risk of death for a given
set of covariates, using regularization when fitting our model. First,
we fit a smooth hazard to the data; for the sake of this example, we
opted for a linear term for time. As casebase makes use of the glmnet
package, we interact with the fitsmoothhazard.fit function using a
matrix interface, where y contains the time and event variables and x
contains all other variables we would like to include in the model. For
the SUPPORT dataset, all factor variables were converted into dummy
variables. Here, we used lasso penalization by setting alpha to 1.

\begin{CodeChunk}

\begin{CodeInput}
R> xInteractions=x*y[,2]
R> colnames(xInteractions) <- paste("d.time*", colnames(xInteractions), sep = "")
R> xFull=cbind(x,xInteractions)
R> cb.ModelInter=casebase::fitSmoothHazard.fit(xFull,y,family="glmnet",time="d.time",event="death",
R>                                        formula_time = ~d.time,alpha=1,ratio=rat)
\end{CodeInput}
\end{CodeChunk}

Then, we take our model and use the absoluteRisk function to integrate
and retrieve our absolute risk function.

\begin{CodeChunk}

\begin{CodeInput}
R> #Estimating the absolute risk curve using the newData parameter.
R> casebaseAbsoluteInter=casebase::absoluteRisk(cb.ModelInter,time = seq(1,max(completeData$d.time), 400),newdata = xFull , s="lambda.1se",method=c("numerical"))
R> 
R> cabI=data.frame(casebaseAbsoluteInter[,1],rowMeans(casebaseAbsoluteInter[,-c(1)]))
\end{CodeInput}
\end{CodeChunk}

We will compare this curve to a regularized version of cox regression,
and a kaplan-meier survival curve. The cox regression absolute risk
curve with lasso penalization required two extra steps to retrieve.
Coxnet from the glmnet package has tools to fit a regularized cox model,
but requires a survival object skeleton of the selected variables, so
that the survival package tools can be used to retrive the absolute
risk. We handle this by fitting a regularized cox model with the glmnet
package, and a cox model from the survival package with the selected
variables from the regularized fit. This second model serves as a
skeleton that permits the use of survival package functions. The
coefficients from the regularized model will replace the coefficients in
our skeleton. the resulting model will have the correct coefficients
when integrating, but will have an incorrect standard error. For the
purposes of this case study, we are only interested in the absolute risk
curve itself and not the standard error.

\begin{CodeChunk}

\begin{CodeInput}
R> #Create u and xCox, which will be used when fitting Cox with glmnet.
R> u=survival::Surv(time = as.numeric(y[,2]), event = as.factor(y[,1]))
R> 
R> xCoxI=as.matrix(cbind(data.frame("(Intercept)"=rep(1,each=690)),xFull))
R> colnames(xCoxI)[1]="(Intercept)"
R> 
R> #hazard for cox using glmnet
R> coxNetI=glmnet::cv.glmnet(x=xCoxI,y=u, family="cox",alpha=1)
R> #convergence demonstrated in plot
R> #plot(coxNet)
R> #taking the coefficient estimates for later use
R> nonzero_covariate_cox <- predict(coxNetI, type = "nonzero", s = "lambda.1se")
R> nonzero_coef_cox <- coef(coxNetI, s = "lambda.1se")
R> #creating a new dataset that only contains the covariates chosen through glmnet.
R> #cleanCoxData<- as.data.frame(cbind(as.numeric(workingData$d.time),as.factor(workingData$death), xCox[,nonzero_covariate_cox$X1]))
R> cleanCoxDataI<-as.data.frame(cbind(as.numeric(y[,2]),as.numeric(y[,1]),xCoxI[,nonzero_covariate_cox$X1]))
R> #newDataCox<-xCox[sam,nonzero_covariate_cox$X1]
R> #fitting a cox model using regular estimation, however we will not keep it.
R> #this is used more as an object place holder.
R> coxNetFitI <- survival::coxph(Surv(time=V1,event=V2) ~ ., data = cleanCoxDataI)
R> #The coefficients of this object will be replaced with the estimates from coxNet.
R> #Doing so makes it so that everything is invalid aside from the coefficients.
R> #In this case, all we need to estimate the absolute risk is the coefficients.
R> #Std. error would be incorrect here, if we were to draw error bars.
R> coxNetFitI$coefficients<-nonzero_coef_cox@x
R> #Fitting absolute risk curve for cox+glmnet. -1 to remove intercept
R> #posCox=nonzero_covariate_cox$X1
R> #newDataCoxI=cleanCoxDataI[,posCox]
R> 
R> abcoxNetI<-survival::survfit(coxNetFitI,type="breslow",newdata=cleanCoxDataI)
R> 
R> abCNI<-abcoxNetI
R> abCNI$surv=rowMeans(abCNI$surv)
\end{CodeInput}
\end{CodeChunk}

\begin{CodeChunk}

\begin{CodeInput}
R> fullDataCox<-as.data.frame(cbind(as.numeric(y[,2]),as.numeric(y[,1]),xCoxI[,-c(1)]))
R> coxFitI <- survival::coxph(Surv(time=V1,event=V2) ~ ., data = fullDataCox)
R> abcoxFitI<-survival::survfit(coxFitI,newdata=fullDataCox)
R> abCFI<-abcoxFitI
R> abCFI$surv<-rowMeans(abCFI$surv)
R> #cb.ModelNorm=casebase::fitSmoothHazard(V2~.,time="V1",event="V2",
R>        #                                ratio=rat,data=fullDataCox)
R> #summary(cb.ModelNorm)#need to figure out how to deal with NAs
R> #casebaseAbsoluteNorm=casebase::absoluteRisk(cb.ModelNorm,time = seq(1,max(completeData$d.time), 200),newdata = newData , s="lambda.1se",method=c("numerical"))
R> #cabn=data.frame(casebaseAbsoluteNorm[,1],rowMeans(casebaseAbsoluteNorm[,-c(1)]))
R> 
\end{CodeInput}
\end{CodeChunk}

We used the survival package to calculate the kaplan-meier absolute risk
function.

\begin{CodeChunk}

\begin{CodeInput}
R> 
R> coxFitup <- survival::coxph(Surv(time=V1,event=V2) ~ ., data = Cox)
R> abcoxFitup<-survival::survfit(coxFitup,newdata=Cox)
R> 
R> #creating a surv object to be used to fit an unadjusted absolute risk curve.
R> # (Kaplan-Meier risk curve)
R> km <- survival::Surv(time = completeData$d.time, event = completeData$death)
R> abKm<-survival::survfit(km~1,type='kaplan-meier',conf.type='log')
\end{CodeInput}
\end{CodeChunk}

Now that our four cumulative incidence estimations have been calculated,
we compare them all in Figure \ref{fig:abSC}. Both coxnet and casebase
decrease the absolute risk by the end of the study, in comparison to
kaplan-meier.

\begin{CodeChunk}

\begin{CodeInput}
R> 
R> #get covariates excluding d.time
R> estimatescb=setDT(as.data.frame(coef(cb.Model)[-c(1,2),1]), keep.rownames = TRUE)[]
R> estimatescb$Model="casebase"
R> colnames(estimatescb)<-c("Coefs","Estimate","Model")
R> estimatescox=setDT(as.data.frame(coef(coxNet)[-c(1),1]), keep.rownames = TRUE)[]
R> estimatescox$Model="Cox"
R> colnames(estimatescox)<-c("Coefs","Estimate","Model")
R> estimates=rbind(estimatescb, estimatescox)
R> estimates$Estimate[estimates$Estimate==0]=NA# make it so unchosen covariates appear empty
R> library(ggstance)# for vertical dodge
R>   ggplot(estimates, aes(x = Estimate, y = Coefs, color = Model)) +
R>         geom_segment(aes(x = 0, y = Coefs, xend = Estimate, yend = Coefs),size=0.5 ) +
R>         geom_point(position=ggstance::position_dodgev(height=0.2))
R> 
\end{CodeInput}
\end{CodeChunk}

\begin{CodeChunk}

\begin{CodeInput}
R> #get covariates excluding d.time
R> library(data.table)
R> estimatescb=setDT(as.data.frame(coef(cb.ModelInter)[-c(1,2),1]), keep.rownames = TRUE)[]
R> estimatescb$Model="casebase"
R> colnames(estimatescb)<-c("Coefs","Estimate","Model")
R> estimatescox=setDT(as.data.frame(coef(coxNetI)[-c(1),1]), keep.rownames = TRUE)[]
R> estimatescox$Model="Cox"
R> colnames(estimatescox)<-c("Coefs","Estimate","Model")
R> estimatesI=rbind(estimatescb, estimatescox)
R> estimatesI$Estimate[estimatesI$Estimate==0]=NA# make it so unchosen covariates appear empty
R> library(ggstance)# for vertical dodge
R>   ggplot(estimatesI, aes(x = Coefs, y = Estimate , color = Model)) +
R>         geom_segment(aes(y = 0, x = Coefs, yend = Estimate, xend = Coefs),size=0.5, position = position_dodge(width = 0.5)) +
R>         geom_point(position = position_dodge(width = 0.5))
R>   
R>     
\end{CodeInput}
\end{CodeChunk}

\hypertarget{case-study-4stanford-heart-transplant-data}{%
\section{Case study 4--Stanford Heart Transplant
Data}\label{case-study-4stanford-heart-transplant-data}}

In the previous case studies, we only considered covariates that were
fixed at baseline. In this next case study, we will use the Stanford
Heart Transplant data
\citep[\citet{crowley1977covariance}]{clark1971cardiac} to show how
case-base sampling can also be used in the context of time-varying
covariates. As an example that already appeared in the literature,
case-base sampling was used to study vaccination safety, where the
exposure period was defined as the week following vaccination
\citep{saarela2015case}. Hence, the main covariate of interest,
i.e.~exposure to the vaccine, was changing over time. In this context,
case-base sampling offers an efficient alternative to nested
case-control designs or self-matching.

Recall the setting of Stanford Heart Transplant study: patients were
admitted to the Stanford program after meeting with their physician and
determining that they were unlikely to respond to other forms of
treatment. After enrollment, the program searched for a suitable donor
for the patient, which could take anywhere between a few days to almost
a year. We are interested in the effect of a heart transplant on
survival; therefore, the patient is considered exposed only after the
transplant has occurred.

As before, we can look at the population-time plot for a graphical
summary of the event incidence. As we can see, most events occur early
during the follow-up period, and therefore we do not expect the hazard
to be constant.

\begin{CodeChunk}

\begin{CodeInput}
R> library(survival)
R> library(casebase)
R> 
R> stanford_popTime <- popTime(jasa, time = "futime", 
R>                             event = "fustat")
R> plot(stanford_popTime)
\end{CodeInput}
\end{CodeChunk}

Since the exposure is time-dependent, we need to manually define the
exposure variable \emph{after} case-base sampling and \emph{before}
fitting the hazard function. For this reason, we will use the
\texttt{sampleCaseBase} function directly.

\begin{CodeChunk}

\begin{CodeInput}
R> library(tidyverse)
R> library(lubridate)
R> 
R> cb_data <- sampleCaseBase(jasa, time = "futime", 
R>                           event = "fustat", ratio = 10)
\end{CodeInput}
\end{CodeChunk}

Next, we will compute the number of days from acceptance into the
program to transplant, and we use this variable to determine whether
each population-moment is exposed or not.

\begin{CodeChunk}

\begin{CodeInput}
R> # Define exposure variable
R> cb_data <- mutate(cb_data,
R>                   txtime = time_length(accept.dt %--% tx.date, 
R>                                        unit = "days"),
R>                   exposure = case_when(
R>                     is.na(txtime) ~ 0L,
R>                     txtime > futime ~ 0L,
R>                     txtime <= futime ~ 1L
R>                   ))
\end{CodeInput}
\end{CodeChunk}

Finally, we can fit the hazard using various linear predictors.

\begin{CodeChunk}

\begin{CodeInput}
R> library(splines)
R> # Fit several models
R> fit1 <- fitSmoothHazard(fustat ~ exposure,
R>                         data = cb_data, time = "futime")
R> fit2 <- fitSmoothHazard(fustat ~ exposure + futime,
R>                         data = cb_data, time = "futime")
R> fit3 <- fitSmoothHazard(fustat ~ exposure + bs(futime),
R>                         data = cb_data, time = "futime")
R> fit4 <- fitSmoothHazard(fustat ~ exposure*bs(futime),
R>                         data = cb_data, time = "futime")
\end{CodeInput}
\end{CodeChunk}

Note that the fourth model (i.e. \texttt{fit4}) includes an interaction
term between exposure and follow-up time. In other words, this model no
longer exhibit proportional hazards. The evidence of non-proportionality
of hazards in the Stanford Heart Transplant data has been widely
discussed \citep{arjas1988graphical}.

We can then compare the goodness of fit of these four models using the
Akaike Information Criterion (AIC).

\begin{CodeChunk}

\begin{CodeInput}
R> # Compute AIC
R> c("Model1" = AIC(fit1),
R>   "Model2" = AIC(fit2),
R>   "Model3" = AIC(fit3),
R>   "Model4" = AIC(fit4))
\end{CodeInput}
\end{CodeChunk}

As we can, the best fit is the fourth model. By visualizing the hazard
functions for both exposed and unexposed individuals, we can more
clearly see how the hazards are no longer proportional.

\begin{CodeChunk}

\begin{CodeInput}
R> # Compute hazards---
R> # First, create a list of time points for both exposure status
R> hazard_data <- expand.grid(exposure = c(0, 1),
R>                            futime = seq(0, 1000,
R>                                         length.out = 100))
R> # Set the offset to zero
R> hazard_data$offset <- 0 
R> # Use predict to get the fitted values, and exponentiate to 
R> # transform to the right scale
R> hazard_data$hazard = exp(predict(fit4, newdata = hazard_data,
R>                                  type = "link"))
R> # Add labels for plots
R> hazard_data$Status = factor(hazard_data$exposure,
R>                             labels = c("NoTrans", "Trans"))
R> 
R> ggplot(hazard_data, aes(futime, hazard, colour = Status)) +
R>     geom_line() +
R>     theme_minimal() +
R>     theme(legend.position = 'top') +
R>     ylab('Hazard') + xlab('Follow-up time')
\end{CodeInput}
\end{CodeChunk}

The non-proportionality seems to be more pronounced at the beginning of
follow-up than the end. Finally, we can turn these estimates of the
hazard function into estimates of the cumulative incidence functions.

\begin{CodeChunk}

\begin{CodeInput}
R> # Compute absolute risk curves
R> newdata <- data.frame(exposure = c(0, 1))
R> absrisk <- absoluteRisk(fit4, newdata = newdata, 
R>                         time = seq(0, 1000, length.out = 100))
R> 
R> colnames(absrisk) <- c("Time", "NoTrans", "Trans")
R> 
R> # Rearrange the data
R> absrisk <- gather(as.data.frame(absrisk),
R>                   "Status", "Risk", -Time)
R>  
R> ggplot(absrisk, aes(Time, Risk, colour = Status)) +
R>   geom_line() +
R>   theme_minimal() +
R>   theme(legend.position = 'top') +
R>   expand_limits(y = 0.5) +
R>   xlab('Follow-up time') + ylab('Cum. Incidence')
\end{CodeInput}
\end{CodeChunk}

Note that we can easily adapt the code above to the situation where a
patient receives a heart transplant at a point in time of interest, for
example after 30 days.

\begin{CodeChunk}

\begin{CodeInput}
R> # Compute hazards---
R> # First, create a list of time points for both exposure status
R> one_yr_haz <- data.frame(futime = seq(0, 365,
R>                                       length.out = 100))
R> one_yr_haz <- mutate(one_yr_haz,
R>                      offset = 0,
R>                      exposure = if_else(futime < 30, 0, 1))
R> one_yr_haz$hazard = exp(predict(fit4, newdata = one_yr_haz,
R>                                 type = "link"))
R> ggplot(one_yr_haz, aes(futime, hazard)) +
R>   geom_line() +
R>   theme_minimal() +
R>   theme(legend.position = 'top') +
R>   ylab('Hazard') + xlab('Follow-up time') +
R>   geom_vline(xintercept = 30, linetype = 'dashed')
\end{CodeInput}
\end{CodeChunk}

We can then compare the 1-year mortality risk without transplant and
with transplant at 30 days.

\begin{CodeChunk}

\begin{CodeInput}
R> absoluteRisk(fit4, newdata = data.frame(exposure = 0),
R>              time = 365)
R> 
R> # Use the trapezoidal rule to estimate the cumulative hazard
R> cumhaz_est <- pracma::trapz(one_yr_haz$futime, one_yr_haz$hazard)
R> 1 - exp(-cumhaz_est)
\end{CodeInput}
\end{CodeChunk}

As we can see, the risk estimate at 1-year is about 30\% lower if the
patient receives a heart transplant at 30 days.

\hypertarget{discussion}{%
\section{Discussion}\label{discussion}}

In this article, we presented the \proglang{R} package \pkg{casebase},
which provides functions for fitting smooth parametric hazards and
estimating cumulative incidence functions using case-base sampling. We
outlined the theoretical underpinnings of the approach, we provided
details about our implementation, and we illustrated the merits of the
approach and the package through four case studies.

As a methodological framework, case-base sampling is very flexible. This
flexibility has been explored before in the literature: for example,
Saarela and Hanley \citeyearpar{saarela2015case} used case-base sampling
to model a time-dependent exposure variable in a vaccine safety study.
As another example, Saarela and Arjas \citeyearpar{saarela2015non}
combined case-base sampling and a Bayesian non-parametric framework to
compute individualized risk assessments for chronic diseases. In the
case studies above, we explored this flexibility along two fronts. On
the one hand, we showed how splines could be used as part of the linear
predictor to model the effect of time on the hazard. This strategy
yielded estimates of the survival function that were qualitatively
similar to semiparametric estimates derived from Cox regression;
however, case-base sampling led to estimates of the survival function
that \emph{vary smoothly in time}. On the other hand, we also displayed
the flexibility of case-base sampling by showing how it could be
combined with penalized logistic regression to perform variable
selection. Even though we did not illustrate it in this article,
case-base sampling can also be combined with the framework of
\emph{generalized additive models}. This functionality has been
implemented in our package \pkg{casebase}. Similarly, case-base sampling
can also be combined with quasi-likelihood estimation to fit survival
models that can account for the presence of over-dispersion. All of
these examples illustrate how the case-base sampling framework in
general, and the package \pkg{casebase} in particular, allows the user
to fit a broad and flexible family of survival functions.

As presented in Hanley \& Miettinen \citeyearpar{hanley2009fitting},
case-base sampling is comprised of three steps: 1) sampling a case
series and a base series from the study; 2) fit the log-hazard as a
linear function of predictors (including time); and 3) use the fitted
hazard to estimate the cumulative incidence curve. Accordingly, our
package provides functions for each step. Moreover, the simple interface
of the \code{fittingSmoothHazard} function resembles the \code{glm}
interface. This interface should look familiar to new users. Our modular
approach also provides a convenient way to extend our package for new
sampling or fitting strategies.

In the case studies above, we compared the performance of case-base
sampling with that of Cox regression and Fine-Gray models. In terms of
function interface, \pkg{casebase} uses a formula interface that is
closer to that of \code{glm}, in that the event variable is the only
variable appearing on the left-hand side of the formula. By contrast,
both \code{survival::coxph} and \code{timereg::comp.risk} use arrays
that captures both the event type and time. Both approaches to modeling
yield user-friendly code. However, in terms of output, both approaches
differ significantly. Case-base sampling produces smooth hazards and
smooth cumulative incidence curves, whereas Cox regression and Fine-Gray
models produce step-wise cumulative incidence functions and never
explicitly model the hazard function. Qualitatively, we showed that by
using splines in the linear predictor, all three models yielded similar
curves. However, the smooth nature of the output of \pkg{casebase}
provides a more intuitive interpretation for consumers of these
predictions. In Table \ref{tab:compCBvsCox}, we provide a side-by-side
comparison between the Cox model and case-base sampling.

\begin{table}
\caption{\label{tab:compCBvsCox}Comparison between the Cox model and case-base sampling}
\centering
\begin{tabular}[t]{llp{5cm}}
\toprule
Feature & Cox model & Case-base sampling\\
\midrule
Model type & Semi-parametric & Fully parametric\\
Time & Left hand side of the formula & Right hand side (allows flexible modeling of time)\\
Cumulative incidence & Step function & Smooth-in-time curve\\
Non-proportional hazards & Interaction of covariates with time & Interaction of covariates with time\\
Model testing &  & Use GLM framework \newline (e.g.\ LRT, AIC, BIC)\\
\addlinespace
Competing risks & Difficult & Cause-specific CIFs\\
% Prediction & Kaplan-Meier-based & ROC, AUC, risk reclassification probabilities\\
\bottomrule
\end{tabular}
\end{table}

Our choice of modeling the log-hazard as a linear function of covariates
allows us to develop a simple computational scheme for estimation.
However, as a downside, it does not allow us to model location and scale
parameters separately like the package \pkg{flexsurv}. For example, if
we look at the Weibull distribution as parametrised in
\code{stats::pweibull}, the log-hazard function is given by
\[ \log h(t; \alpha, \beta) = \left[\log(\alpha/\beta) - (\alpha - 1)\log(\beta)\right] + (\alpha - 1)\log t,\]
where \(\alpha,\beta\) are shape and scale parameters, respectively.
Unlike \pkg{casebase}, the approach taken by \pkg{flexsurv} allows the
user to model the scale parameter as a function of covariates. Of
course, this added flexibility comes at the cost of interpretability: by
modeling the log-hazard directly, the parameter estimates from
\pkg{casebase} can be interpreted as estimates of log-hazard ratios. To
improve the flexibility of \pkg{casebase} at capturing the scale of a
parametric family, we could replace the logistic regression with its
quasi-likelihood counterpart and therefore model over- and
under-dispersion with respect to the logistic likelihood. We defer the
study of the properties and performance of such a model to a future
article.

Future work will look at some of the methodological extensions of
case-base sampling. First, to assess the quality of the model fit, we
want to study the properties of the residuals (e.g.~Cox-Snell,
martingale). More work needs to be done to understand these residuals in
the context of the partial likelihood underlying case-base sampling. The
resulting diagnostic tools would then be integrated in this package.
Also, we are interested in extending case-base sampling to account for
interval censoring. This type of censoring is very common in
longitudinal studies, and many packages (e.g. \pkg{SmoothHazard},
\pkg{survival} and \pkg{Rstpm2}) provide functions to account for it.
Again, we hope to include any resulting methodology as part of this
package.

In future versions of the package, we also want to increase the
complement of diagnostic and inferential tools that are currently
available. For example, we would like to include the ability to compute
confidence intervals for the cumulative incidence curve. The delta
method or parametric bootstrap are two different strategies we can use
to construct approximate confidence intervals. Furthermore, we would
like to include functions to compute calibration and discrimination
statistics (e.g.~Brier score, AUC) for our models. Saarela and Arjas
\citeyearpar{saarela2015non} also describe how to obtain a posterior
distribution for the AUC from their model. Their approach could also be
included in \pkg{casebase}. Finally, we want to provide more flexibility
in how the case-base sampling is performed. This could be achieved by
adding a \code{hazard} argument to the function \code{sampleCaseBase}.
In this way, users could specify their own sampling mechanism. For
example, they could provide a hazard that gives sampling probabilities
that are proportional to the cardiovascular disease event rate given by
the Framingham score \citep{saarela2015non}.

In conclusion, we presented the \proglang{R} package \pkg{casebase}
which implements case-base sampling for fitting parametric survival
models and for estimating smooth cumulative incidence functions using
the framework of generalized linear models. We strongly believe that its
flexibility and its foundation on the familiar logistic regression model
will make it appealing to new and established practioners.

\hypertarget{environment-details}{%
\section{Environment Details}\label{environment-details}}

This report was generated on 2020-03-03 13:15:36 using the following
computational environment and dependencies:

\begin{CodeChunk}

\begin{CodeInput}
R> # which R packages and versions?
R> # devtools::session_info()
R> sessionInfo()
\end{CodeInput}

\begin{CodeOutput}
#> R version 3.6.2 (2019-12-12)
#> Platform: x86_64-pc-linux-gnu (64-bit)
#> Running under: Ubuntu 18.04.4 LTS
#> 
#> Matrix products: default
#> BLAS:   /usr/lib/x86_64-linux-gnu/openblas/libblas.so.3
#> LAPACK: /usr/lib/x86_64-linux-gnu/libopenblasp-r0.2.20.so
#> 
#> locale:
#>  [1] LC_CTYPE=en_CA.UTF-8       LC_NUMERIC=C              
#>  [3] LC_TIME=en_CA.UTF-8        LC_COLLATE=en_CA.UTF-8    
#>  [5] LC_MONETARY=en_CA.UTF-8    LC_MESSAGES=en_CA.UTF-8   
#>  [7] LC_PAPER=en_CA.UTF-8       LC_NAME=C                 
#>  [9] LC_ADDRESS=C               LC_TELEPHONE=C            
#> [11] LC_MEASUREMENT=en_CA.UTF-8 LC_IDENTIFICATION=C       
#> 
#> attached base packages:
#> [1] splines   stats     graphics  grDevices utils     datasets  methods  
#> [8] base     
#> 
#> other attached packages:
#>  [1] pracma_2.2.9        kableExtra_1.1.0    data.table_1.12.8  
#>  [4] glmnet_3.0-1        Matrix_1.2-18       casebase_0.2.1.9001
#>  [7] survival_3.1-8      magrittr_1.5        forcats_0.4.0      
#> [10] stringr_1.4.0       dplyr_0.8.3         purrr_0.3.3        
#> [13] readr_1.3.1         tidyr_1.0.0         tibble_2.1.3       
#> [16] ggplot2_3.2.1       tidyverse_1.3.0    
#> 
#> loaded via a namespace (and not attached):
#>  [1] Rcpp_1.0.3        lubridate_1.7.4   lattice_0.20-40   assertthat_0.2.1 
#>  [5] digest_0.6.25     foreach_1.4.7     R6_2.4.1          cellranger_1.1.0 
#>  [9] backports_1.1.5   reprex_0.3.0      stats4_3.6.2      evaluate_0.14    
#> [13] httr_1.4.1        pillar_1.4.3      rlang_0.4.5       lazyeval_0.2.2   
#> [17] readxl_1.3.1      rstudioapi_0.10   rticles_0.12      rmarkdown_1.18   
#> [21] webshot_0.5.2     munsell_0.5.0     broom_0.5.2       compiler_3.6.2   
#> [25] modelr_0.1.5      xfun_0.11         pkgconfig_2.0.3   shape_1.4.4      
#> [29] mgcv_1.8-31       htmltools_0.4.0   tidyselect_0.2.5  codetools_0.2-16 
#> [33] viridisLite_0.3.0 fansi_0.4.1       crayon_1.3.4      dbplyr_1.4.2     
#> [37] withr_2.1.2       grid_3.6.2        nlme_3.1-144      jsonlite_1.6.1   
#> [41] gtable_0.3.0      lifecycle_0.1.0   DBI_1.0.0         scales_1.1.0     
#> [45] cli_2.0.2         stringi_1.4.6     fs_1.3.1          xml2_1.2.2       
#> [49] generics_0.0.2    vctrs_0.2.3       iterators_1.0.12  tools_3.6.2      
#> [53] glue_1.3.1        hms_0.5.2         yaml_2.2.1        colorspace_1.4-1 
#> [57] rvest_0.3.5       VGAM_1.1-2        knitr_1.26        haven_2.2.0
\end{CodeOutput}
\end{CodeChunk}

The current Git commit details are:

\begin{CodeChunk}

\begin{CodeInput}
R> # what commit is this file at? 
R> git2r::repository(here::here())
\end{CodeInput}

\begin{CodeOutput}
#> Local:    review /home/turgeonm/Documents/git_repos/cbpaper
#> Head:     [8804f68] 2020-02-21: Merge branch 'Jesse-Islam-master'
\end{CodeOutput}
\end{CodeChunk}

\bibliography{references.bib}


\end{document}

